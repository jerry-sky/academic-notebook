\documentclass[14pt]{article}
% \usepackage{polski}
\usepackage[english]{babel}
\usepackage[utf8]{inputenc}
\usepackage{amsmath}
\usepackage{amsfonts}
\usepackage{xcolor}
\usepackage{graphicx}
\graphicspath{ {./img/} }
\usepackage{shadowtext}
\usepackage{hyperref}
\hypersetup{%
  colorlinks=false,% hyperlinks will be black
  linkbordercolor=red,% hyperlink borders will be red
  pdfborderstyle={/S/U/W 1}% border style will be underline of width 1pt
}
\usepackage[margin=2.5cm]{geometry}
\usepackage{algpseudocode}
\usepackage{algorithm}

%Import the natbib package and sets a bibliography  and citation styles
\usepackage[numbers]{natbib}
\bibliographystyle{plainnat}

\linespread{1.3}

\title{TabuSearch: Job-shop scheduling problem}
\author{Jerzy Wroczyński}
\date{2020-06-04}

\begin{document}

\maketitle

\section{Introduction}

The job shop scheduling problem is one of many theoretic scheduling problems. In a paper by \citet{amico-trubian} it was classified as $J || C_{\max}$ using the notation introduced by \citet{graham}. Letter $J$ represents „job shop scheduling problem”, two vertical lines with nothing in between mean no further job characteristics are given and $C_{\max}$ defines the optimization problem as minimizing the maximum completion time of all given jobs.

Of course, there are many different types of such problems e.g. there can be a predetermined quantity of machines e.g. only one machine, jobs can have certain characteristics e.g. each job has a \textit{fuzzy due date} etc. but in this paper the problem classified in the previous paragraph will be examined.

\hspace{2pt}

We are given the following resources:
\begin{enumerate}
  \item a set $J$ of $n$ jobs to schedule,
  \item a set $O = \{1,\dots,N\}$ of $N$ atomic operations
  \item a set $M$ of $m$ machines.
\end{enumerate}

For each job $J_j$ there is a sequence of operations $O_{i,j} \in O$ and each of these operations has to be processed without interruption separately on machine $\mu_{i,j} \in M$ for $d_{i,j}$ units of time using already mentioned notation \cite{graham}.

For better understanding of a such schedule problem a visual aid of a Gantt chart can be used:
\begin{figure}
  \centering

  \caption{Example Gantt chart}
  \label{example-gantt}
\end{figure}


\bibliography{bib}

\end{document}
