\documentclass[14pt]{article}
\usepackage{polski}
\usepackage[utf8]{inputenc}
\usepackage{amsmath}
\usepackage{amsfonts}
\usepackage{tabto,lipsum}
\usepackage{xcolor}
\usepackage{shadowtext}
\usepackage{hyperref}
\hypersetup{%
  colorlinks=false,% hyperlinks will be black
  linkbordercolor=red,% hyperlink borders will be red
  pdfborderstyle={/S/U/W 1}% border style will be underline of width 1pt
}
\usepackage[margin=3cm]{geometry}
\usepackage{algpseudocode}
\usepackage{algorithm}

\PassOptionsToPackage{usenames,dvipsnames,svgnames}{xcolor}
\usepackage{tikz}
\usetikzlibrary{arrows,positioning,automata}

\linespread{1.3}

\title{Lista 7}
\author{Zadanie 3}
\date{------------}

\begin{document}

\maketitle

\section{Problem}

Należy zaimplementować kolejkę priorytetową na bazie Binary Search Tree.

\section{Concept}

Musimy odpowiednio zaimplementować wszystkie operacje, które można wykonać na kolejce priorytetowej:
\begin{enumerate}
  \item \texttt{Insert}
  \item \texttt{Minimum}
  \item \texttt{ExtractMin}
  \item \texttt{DecreaseKey}
  \item \texttt{Union}
  \item \texttt{Delete}
\end{enumerate}

W celu osiągnięcia jak najlepszej złożoności obliczeniowej użyjemy tutaj drzew czerwono-czarnych, które są rozszerzeniem drzew BST.

\section{Rozwiązanie}

W przypadku operacji \texttt{Insert}, \texttt{Minimum}, \texttt{Delete} wykorzystujemy standardowe operacje o takiej samej nazwie, które już są w RB-Tree. Złożoność obliczeniowa tych procedur to $O(\lg n)$.

Operacja \texttt{ExtractMin} działa w taki sam sposób jak \texttt{Minimum}, przy czym dodatkowo wywoływana jest funkcja \texttt{Delete} na tym samym elemencie. Złożoność obliczeniowa wynosi $O(\lg n)$, ponieważ obie te operacje mają taką złożoność.

Operacja \texttt{DecreaseKey} najpierw usuwa zadany element i wstawia go ponownie ze zmniejszonym priorytetem. Złożoność obliczeniowa operacji \texttt{Insert} i \texttt{Delete} wynosi $O(\lg n)$ więc złożoność niniejszej operacji również wynosi $O(\lg n)$.

Operacja \texttt{Union}$(A,B)$ najpierw wykonuje operację \texttt{inorder} na obu drzewach, następnie wykonuje operację \texttt{merge} która scala te dwie posortowane tablice w jedną, a następnie tworzy nowe drzewo przy pomocy poniższego algorytmu:

\begin{algorithm}[H]
  \caption{\texttt{sorted\_array\_to\_RBT}}
  \begin{algorithmic}[1]
    \State $n = \texttt{length}(\texttt{merged})$

    \If{$n=0$}:
      \State \texttt{return}
    \EndIf

    \State $\texttt{middle} \gets \left\lfloor\frac{n}{2}\right\rfloor$
    \State $\texttt{root} \gets \texttt{merged}[\texttt{middle}]$
    \State $\texttt{root}.\mathrm{left} \gets \texttt{sorted\_array\_to\_RBT}(\texttt{merged}[0\dots\texttt{middle}-1])$
    \State $\texttt{root}.\mathrm{right} \gets \texttt{sorted\_array\_to\_RBT}(\texttt{merged}[\texttt{middle}+1\dots n])$
    \State \texttt{return root}
  \end{algorithmic}
\end{algorithm}

Złożoność powyższego algorytmu wynosi $O(n)$, ponieważ wykonujemy $n$ razy operacje o złożoności $O(1)$. Złożoność obliczeniowa obu pozostałych operacji \texttt{inorder} i \texttt{merge} wynosi $O(n)$.
Zatem złożoność operacji \texttt{Union} również wynosi $O(n)$.

Porównanie złożoności obliczeniowych dla dwóch implementacji kolejki priorytetowej:

\begin{center}
  \begin{tabular}{|c|c c|}
  \hline
   & Drzewo binarne (RB-Tree) & Kopiec binarny \\ [0.5ex]
  \hline\hline
  \texttt{Insert} & $O(\lg n)$ & $O(\lg n)$  \\
  \hline
  \texttt{Minimum} & $O(\lg n)$ & $O(1)$ \\
  \hline
  \texttt{ExtractMin} & $O(\lg n)$ & $O(\lg n)$ \\
  \hline
  \texttt{DecreaseKey} & $O(\lg n)$ & $O(\lg n)$ \\
  \hline
  \texttt{Union} & $O(n)$ & $O(n)$ \\
  \hline
  \texttt{Delete} & $O(\lg n)$ & $O(\lg n)$ \\
  \hline
  \end{tabular}
\end{center}

\end{document}
