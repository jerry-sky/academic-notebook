\documentclass[14pt]{article}
\usepackage{polski}
\usepackage[utf8]{inputenc}
\usepackage{amsmath}
\usepackage{amsfonts}
\usepackage{tabto,lipsum}
\usepackage{xcolor}
\usepackage{shadowtext}
\usepackage{hyperref}
\hypersetup{%
  colorlinks=false,% hyperlinks will be black
  linkbordercolor=red,% hyperlink borders will be red
  pdfborderstyle={/S/U/W 1}% border style will be underline of width 1pt
}
\usepackage[margin=2.8cm]{geometry}
\usepackage{algpseudocode}
\usepackage{algorithm}

\PassOptionsToPackage{usenames,dvipsnames,svgnames}{xcolor}
\usepackage{tikz}
\usetikzlibrary{arrows,positioning,automata}

\linespread{1.3}

\title{Lista 8}
\author{Zadanie 6}
\date{------------}

\begin{document}

\maketitle

\section{Problem}

Rozważamy graf skierowany z wagami na krawędziach (możliwe ujemne wagi). Chcemy uzyskać algorytm do wyznaczania najkrótszej ścieżki pomiędzy dowolnymi dwoma wierzchołkami przy czym taka ścieżka nie może być dłuższa niż $k$.

\section{Concept}

Użyjemy tutaj zmodyfikowanej wersji algorytmu Bellmana-Forda. Otóż, w oryginalnej wersji tego algorytmu rozważamy drogi o długości ograniczonej jedynie przez liczbę wierzchołków w danym grafie. Takie ograniczenie dawało nam złożoność obliczeniową $O(|E|\cdot|V|)$ jako, że wszystkie krawędzie skanowane są w najgorszym przypadku $|V| - 1$ razy. Tutaj interesują nas ścieżki o długości maksymalnie $k$.

\section{Rozwiązanie}

W poniższym algorytmie używamy zmodyfikowanej wersji algorytmu Bellmana-Forda. Wprowadzamy dodatkowe ograniczenie w postaci maksymalnej liczby iteracji głównej pętli. (Wierzchołkiem startowym jest $s$.)

\begin{algorithm}[H]
  \caption{Zmodyfikowany algorytm Bellmana-Forda}
  \begin{algorithmic}[1]
    \For{\textbf{all} $v\in V$}:
      \State $v.\mathrm{dist} \gets \infty$
      \State $v.\mathrm{prev} \gets \mathrm{NULL}$
    \EndFor
    \State $s.\mathrm{dist} \gets 0$
    \State $s.\mathrm{prev} \gets s$
    \State $\mathrm{changed} \gets \mathrm{TRUE}$
    \State $i \gets 0$
    \Repeat:
      \State $i \gets i + 1$
      \State $\mathrm{changed} \gets \mathrm{FALSE}$
      \For{\textbf{all} $(u,v) \in E$}:
        \If{$v.\mathrm{dist} > u.\mathrm{dist} + c(u,v)$}:
          \State $v.\mathrm{dist} \gets u.\mathrm{dist} + c(u,v)$
          \State $v.\mathrm{prev} \gets u$
          \State $\mathrm{changed} \gets \mathrm{TRUE}$
        \EndIf
      \EndFor
    \Until{$\mathrm{changed} \equiv \mathrm{FALSE} \lor i = k$}\label{modified-alg-line}
  \end{algorithmic}
\end{algorithm}

W linijce \ref{modified-alg-line} został dodany dodatkowy warunek końcowy pętli. Ogranicza to maksymalną długość ścieżki w grafie oraz zmienia złożoność obliczeniową algorytmu, która teraz wynosi $O(k\cdot |E|)$.

Długość żądanej najkrótszej ścieżki oczywiście mieści się w parametrze „$\mathrm{dist}$” wierzchołka docelowego. Oczywiście może się stać, że ścieżka o liczbie krawędzi $\le k$ nie istnieje. Jeśli obliczona odległość wynosi $\infty$ ścieżka nie została znaleziona.

Żeby uzyskać drogę „wierzchołek po wierzchołku” należy iteracyjnie przejść po wierzchołkach znalezionej drogi używając parametru „$\mathrm{prev}$” aż nie dojdziemy do wierzchołka startowego.


\end{document}
