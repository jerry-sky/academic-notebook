\documentclass[14pt]{article}
\usepackage{polski}
\usepackage[utf8]{inputenc}
\usepackage{amsmath}
\usepackage{amsfonts}
\usepackage{tabto,lipsum}
\usepackage{xcolor}
\usepackage{shadowtext}
\usepackage{hyperref}
\hypersetup{%
  colorlinks=false,% hyperlinks will be black
  linkbordercolor=red,% hyperlink borders will be red
  pdfborderstyle={/S/U/W 1}% border style will be underline of width 1pt
}
\usepackage[margin=3cm]{geometry}

\linespread{1.3}

\title{Lista 5}
\author{Zadanie 3}
\date{------------}

\begin{document}

\maketitle

\section{Rozwiązanie w czasie liniowym}

Wystarczy wziąć strukturę RB-Tree, w której operacje \texttt{Search}, \texttt{Insert} oraz \texttt{Delete} mają złożoność równą $O(\log n)$.

Wówczas operacja \texttt{Min-Luka} działa poprzez generowanie listy elementów w drzewie w kolejności \texttt{inorder} i liniowo przegląda każdą parę elementów, znajdując najmniejszą różnicę. Jeśli znaleziono różnicę równą $0$ zwraca $0$ jako, że to najmniejsza możliwa różnica pomiędzy elementami.
\texttt{Min-Luka} ma wtedy złożoność obliczeniową równą $O(n)$.

\section{Rozwiązanie w stałym czasie}

Weźmy strukturę RB-Tree, i dodajmy do definicji węzła pola:
\begin{enumerate}
  \item \texttt{min} — określające $\min$ z elementów w danym pod-drzewie tworzonym przez dany węzeł
  \item \texttt{max} — określające $\max$ z elementów w danym pod-drzewie tworzonym przez dany węzeł
  \item \texttt{local-min-diff} — określające lokalny żądany wynik operacji \texttt{Min-Luka}
\end{enumerate}

Należy zauważyć, że nowe pola można zdefiniować w następujący sposób:

$\forall_{x \in T}:\\$

$
x.\texttt{min} = \begin{cases}
  x.\texttt{left}.\texttt{min} & \text{jeśli }x.\texttt{left} \neq \mathrm{NIL}\\
  x.\texttt{value} & \text{otherwise}
\end{cases}
$

$\\$

$
x.\texttt{max} = \begin{cases}
  x.\texttt{right}.\texttt{max} & \text{jeśli }x.\texttt{right} \neq \mathrm{NIL}\\
  x.\texttt{value} & \text{otherwise}
\end{cases}
$
$\\$

$\texttt{tmp} = \{\infty\}$

\texttt{if} $x.\texttt{left} \neq \mathrm{NIL}$:

\indent\indent $\texttt{tmp} = \{x.\texttt{left}.\texttt{local-min-diff}, ~x.\texttt{value} - x.\texttt{left}.\texttt{max}\}$

\texttt{if} $x.\texttt{right} \neq \mathrm{NIL}$:

\indent\indent $\texttt{tmp} = \texttt{tmp} \cup \{x.\texttt{right}.\texttt{local-min-diff}, ~x.\texttt{right}.\texttt{min} - x.\texttt{value}\}$

$x.\texttt{local-min-diff} = \min(\texttt{tmp})$

$\\$

Dzięki czemu można łatwo zauważyć, że nowo dodane pola zależą jedynie od wartości węzła lub jego bezpośrednich potomków. Wówczas korzystając z twierdzenia 14.1 z książki „Wprowadzenie do algorytmów” mamy pewność, że złożoność czasowa operacji \texttt{Insert}, \texttt{Delete} oraz \texttt{Search} nie zmieni się.

Oczywiście musimy zmodyfikować standardowe operacje na RB-Tree.

\subsection{\texttt{Insert}}

Jedyne co musimy dodać do standardowej operacji \texttt{Insert} to zapamiętywanie przez węzły wartości minimalnej i maksymalnej dla pod-drzew które tworzą.

Każdy z węzłów startuje z polami \texttt{min} oraz \texttt{max} przyrównanymi do wartości tego węzła. Pole \texttt{local-min-diff} aktualizujemy na podstawie wzoru podanego wyżej.

Kiedy dodajemy nowy węzeł $x$ to dla każdego węzła który napotka węzeł $x$ aktualizujemy wartości pól \texttt{min} oraz \texttt{max} dla tego napotkanego węzła (jeśli wartość $x$ jest większa niż pole \texttt{max} węzła napotkanego to zmieniamy wartość tego pola, \texttt{min} analogicznie).

Przy ewentualnej rotacji aktualizujemy węzły których dotyczy dana rotacja w czasie $O(\log n)$ dlatego, że aktualizujemy wartości pól przodków węzłów rotowanych. Zważając na wysokość drzewa $\le 2\log(n+1)$ nasza złożoność pozostanie $O(\log n)$.

\subsection{\texttt{Delete}}

Przy usuwaniu musimy zaktualizować wartości pól \texttt{min} oraz \texttt{max} przodków węzła usuwanego. Robimy to w czasie $O(\log n)$ dlatego, że RB-Tree ma wysokość $\le 2\log(n+1)$.

\subsection{\texttt{Search}}

Używamy tutaj standardowej operacji \texttt{RB-Search} bez żadnych modyfikacji.

\subsection{\texttt{Min-Luka}}

Zwracamy wartość pola \texttt{local-min-diff} węzła \texttt{root} drzewa. Jest to pojedyncza operacja przetwarzająca jeden węzeł przez co złożoność obliczeniowa tej operacji wynosi $O(1)$.


\end{document}
