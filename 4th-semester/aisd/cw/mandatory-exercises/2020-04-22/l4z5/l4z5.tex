\documentclass[14pt]{article}
\usepackage{polski}
\usepackage[utf8]{inputenc}
\usepackage{amsmath}
\usepackage{amsfonts}
\usepackage{tabto,lipsum}
\usepackage{xcolor}
\usepackage{shadowtext}
\usepackage{hyperref}
\hypersetup{%
  colorlinks=false,% hyperlinks will be black
  linkbordercolor=red,% hyperlink borders will be red
  pdfborderstyle={/S/U/W 1}% border style will be underline of width 1pt
}
\usepackage[margin=3cm]{geometry}

\linespread{1.3}

\title{Lista 4}
\author{Zadanie 5}
\date{---------------------}

\begin{document}

\maketitle

\section{Problem}

Potrzebujemy algorytmu liczącego ilość nieporządków w ciągu $a_1, \dots, a_n$ gdzie nieporządek rozumiemy jako zależność między elementami $a_i, a_j$ taką, że $i < j \land a_i > a_j$.

\section{Concept}

Dla każdego $a_i$ musimy sprawdzić wszystkie $\forall_{j > i} ~a_j$ czy nie zachodzi $a_i > a_j$.

Moglibyśmy tutaj wykorzystać algorytm \texttt{InsertionSort} i zliczać ilość przesunięć każdego z elementów podczas szukania ich właściwej pozycji na sortowanej tablicy. Jednakże, możemy podejść do problemu bezpośrednio i użyć dwóch pętli gdzie jedna z nich jest zagnieżdżona w drugiej.
W obu przypadkach mamy taką samą złożoność obliczeniową z perspektywy $O$. (Akurat tutaj poniższy algorytm ma korzystniejszą stałą $c$ stojącą obok funkcji $n^2$.)

\section{Rozwiązanie}

Algorytm:\\
\indent \texttt{nieporządki $\leftarrow$ 0}

\indent \texttt{dla i = 1,\dots,n:}

\indent\indent \texttt{dla j = (i+1),\dots,n:}

\indent\indent\indent \texttt{jeśli} $a_i > a_j$ \texttt{wówczas:}

\indent\indent\indent\indent \texttt{nieporządki $\leftarrow$ nieporządki + 1}

\indent \texttt{zwróć nieporządki}

\end{document}
